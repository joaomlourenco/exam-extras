% % \edef\POINTSREM{\fpeval{\POINTSTOTAL-\POINTSMC}}
% % % \SUBTRACT{\POINTSTOTAL}{\POINTSMC}{\POINTSREM}
% % \edef\POINTSMCQ{\fpeval{\POINTSMC/\mynumquestionsmcq}}
% % % \DIVIDE{\POINTSMC}{\mynumquestionsmcq}{\POINTSMCQ}
% \SUBTRACT{\mynumquestions}{\mynumquestionsmcq}{\NUMQNOTMC}
% \ifnum\NUMQNOTMC>0
% \ifnum\numparts<1\relax\def\numparts{1}\fi
% \DIVIDE{\POINTSREM}{\numparts}{\POINTSNOTMCQ}
% \else
% \def\POINTSNOTMCQ{0}
% \fi
\def\NQ{
  \ifnum\mynumquestions>1 \mynumquestions\else 1 \fi
}

\def\PPQ{\fpeval{round(20/\NQ,3)}}

\providecommand{\MAXPENALTYAT}{5}
\providecommand{\MAXPENALTYPERCENT}{33}
\edef\PENALTYSTEP{\fpeval{\MAXPENALTYPERCENT/(\MAXPENALTYAT-1)}}
\def\printpenalty{%
  \null\hfill
  \foreach[count=\i] \percentage in {0, \PENALTYSTEP, ..., \MAXPENALTYPERCENT} {
    \edef\mypercent{\fpeval{round(\percentage,2)}}
    $ \ifdim\mypercent pt=\MAXPENALTYPERCENT pt\ge\fi
      \i =
      \fpeval{round(\PPQ*\mypercent/100,3)}~(\mypercent\%)$
  }
  \hfill\null
}


\def\headerenglish{%
\begin{center}
  {\Large Please read these instructions carefully!}
  \begin{tcolorbox}[width=0.9\textwidth]
		\small% \setdefaultleftmargin{1em}{}{}{}{.55em}{.55em}
		\begin{itemize}[itemsep=1pt,leftmargin=0.5cm]
      \itemsep=0pt
      \item Answer the questions on the answer sheet.
      \item You may use the back of the test paper sheets for scratch work.
      \item Do not unstaple the test paper sheets!
      \item Instructions for answering: 
      \newcommand{\CC}{\textcolor{black!60}{\CIRCLE}}
      \newcommand{\cc}{\textcolor{Black!60}{\Circle}}
      \begin{tabular}[t]{lccccc}
        & ~A & B & C & D & E~~\\
            Select the answer (A):                        
                & (\CC    & \cc   & \cc   & \cc   & \cc)\\
            Replace the answer (A) with answer (C):       
                & (\rlap{\CC}{\kern-1pt$\bigtimes$}
                          & \cc   & \CC   & \cc   & \cc)\\
            Cancel (C) and reactivate the answer (A):     
                & (\rlap{\raisebox{-1.5pt}{\kern-2.35pt\LARGE\Circle}}%
                      {\rlap{\CC}{\kern-1pt$\bigtimes$}} 
                          & \cc   & \rlap{\CC}{\kern-1pt$\bigtimes$}
                                          & \cc   & \cc)\\
            Do not answer (leave all circles unfilled):   
                & (\cc    & \cc    & \cc   & \cc  & \cc)\\
            Do not answer (two or more answers crossed):  
                & (\rlap{\CC}{\kern-1pt$\bigtimes$} 
                          & \cc    & \cc   & \rlap{\CC}{\kern-1pt$\bigtimes$} 
                                                  & \cc)\\
            \textbf{NEVER leave a single answer crossed}: 
                & \raisebox{1.5pt}{\makebox[0pt][l]{\kern-3pt\rule{3.7cm}{1pt}}}%
                  (\cc    & \rlap{\CC}{\kern-1pt$\bigtimes$} 
                                   & \cc   & \cc  & \cc)\\
      \end{tabular}
      \item This \MakeLowercase{\TESTTYPE} has~\numquestions\ questions, each question is worth \PPQ\ points.
      \item \textbf{PENALTY POINTS} for incorrect answers:\\%\hspace*{4em}
      \printpenalty
		\end{itemize}
  \end{tcolorbox}
\end{center}
% \smallskip
\begin{tcolorbox}
  \rule{0pt}{4ex}%
  \textbf{NAME:} \hrulefill\hrulefill\hrulefill\quad\textbf{Number:} \hrulefill
\end{tcolorbox}
}



\newcommand\headerportuguese{%
\begin{center}
  {\Large Por favor leia estas instruções com atenção!}
  \begin{tcolorbox}[width=0.9\textwidth]
		\small%\setdefaultleftmargin{1em}{}{}{}{.55em}{.55em}
		\begin{itemize}[itemsep=1pt,leftmargin=0.5cm]
      \item Responda às questões na folha de resposta.
      \item Pode usar as costas do enunciado para rascunho.
      \item Não desagrafe o enunciado!
      \item Instruções para responder: 
      \newcommand{\C}{\textcolor{black!60}{\CIRCLE}}
      \newcommand{\Cc}{\Textcolor{Black!60}{\Circle}}
      \begin{tabular}[t]{lccccc}
        & ~A & B & C & D & E~~\\
            Seleccionar a resposta (A):                        
                & (\CC    & \cc   & \cc   & \cc   & \cc)\\
            Substituir a resposta (A) pela resposta (C):     
                & (\rlap{\CC}{\kern-1pt$\bigtimes$}
                          & \cc   & \CC   & \cc   & \cc)\\
            Cancelar (C) e reactivar a resposta (A):   
                & (\rlap{\raisebox{-1.5pt}{\kern-2.35pt\LARGE\Circle}}%
                      {\rlap{\CC}{\kern-1pt$\bigtimes$}} 
                          & \cc   & \rlap{\CC}{\kern-1pt$\bigtimes$}
                                          & \cc   & \cc)\\
            Não responder (deixar todas em branco):
                & (\cc    & \cc    & \cc   & \cc  & \cc)\\
            Não responder (duas respostas quaisquer cruzadas): 
                & (\rlap{\CC}{\kern-1pt$\bigtimes$} 
                          & \cc    & \cc   & \rlap{\CC}{\kern-1pt$\bigtimes$} 
                                                  & \cc)\\
            \textbf{NUNCA deixar uma resposta cruzada solitária}: 
                & \raisebox{1.5pt}{\makebox[0pt][l]{\kern-3pt\rule{3.7cm}{1pt}}}%
                  (\cc    & \rlap{\CC}{\kern-1pt$\bigtimes$} 
                                   & \cc   & \cc  & \cc)\\
      \end{tabular}
      \item Este \MakeLowercase{\TESTTYPE} tem~\numquestions\ questões, cada questão tem uma cotação de \fpeval{round(20/\NQ,3)}\ pontos.
      \item \textbf{DESCONTO} por respostas erradas (em percentagem da cotação da respectiva questão):\\%
      \printpenalty
		\end{itemize}
  \end{tcolorbox}
\end{center}
% \smallskip
\begin{tcolorbox}
  \rule{0pt}{4ex}%
  \textbf{Nome:} \hrulefill\hrulefill\hrulefill\quad\textbf{Número:} \hrulefill
\end{tcolorbox}
}

% \thispagestyle{plain}


\newcommand\headerportuges{\headerportuguese}
\newcommand\headerportugues{\headerportuguese}


\csname header\languagename\endcsname
